% --------------------------------------------------------------
% This is all preamble stuff that you don't have to worry about.
% Head down to where it says "Start here"
% --------------------------------------------------------------
 
\documentclass[12pt]{article}
 
\usepackage[margin=1in]{geometry} 
\usepackage{amsmath,amsthm,amssymb,scrextend}
\usepackage{fancyhdr}
\pagestyle{fancy}

 
\newcommand{\N}{\mathbb{N}}
\newcommand{\Z}{\mathbb{Z}}
\newcommand{\I}{\mathbb{I}}
\newcommand{\R}{\mathbb{R}}
\newcommand{\Q}{\mathbb{Q}}
\renewcommand{\qed}{\hfill$\blacksquare$}
\let\newproof\proof
\renewenvironment{proof}{\begin{addmargin}[1em]{0em}\begin{newproof}}{\end{newproof}\end{addmargin}\qed}
% \newcommand{\expl}[1]{\text{\hfill[#1]}$}
 
\newenvironment{theorem}[2][Theorem]{\begin{trivlist}
    \item[\hskip \labelsep {\bfseries #1}\hskip \labelsep {\bfseries #2.}]}{\end{trivlist}}
    \newenvironment{lemma}[2][Lemma]{\begin{trivlist}
        \item[\hskip \labelsep {\bfseries #1}\hskip \labelsep {\bfseries #2.}]}{\end{trivlist}}
        \newenvironment{problem}[2][Problem]{\begin{trivlist}
            \item[\hskip \labelsep {\bfseries #1}\hskip \labelsep {\bfseries #2.}]}{\end{trivlist}}
            \newenvironment{exercise}[2][Exercise]{\begin{trivlist}
                \item[\hskip \labelsep {\bfseries #1}\hskip \labelsep {\bfseries #2.}]}{\end{trivlist}}
                \newenvironment{reflection}[2][Reflection]{\begin{trivlist}
                    \item[\hskip \labelsep {\bfseries #1}\hskip \labelsep {\bfseries #2.}]}{\end{trivlist}}
                    \newenvironment{proposition}[2][Proposition]{\begin{trivlist}
                        \item[\hskip \labelsep {\bfseries #1}\hskip \labelsep {\bfseries #2.}]}{\end{trivlist}}
                        \newenvironment{corollary}[2][Corollary]{\begin{trivlist}
                            \item[\hskip \labelsep {\bfseries #1}\hskip \labelsep {\bfseries #2.}]}{\end{trivlist}}
                             
                            \begin{document}
                             
                            % --------------------------------------------------------------
                            %                         Start here
                            % --------------------------------------------------------------
                            
                            \lhead{Math 250}
                            \chead{Luis Gascon}
                            \rhead{Chapter 4.2}
                             
                            \begin{problem}{2}
                              $\newline$
                              \textbf{Theorem:} For every integer $m$, if $m$ is even then $3m+5$ is odd
                            \end{problem}
                            \begin{proof}
                              Let $n \in \Z \ni n$ is odd \\
                              By definition of odd, $k = 2$
                              \begin{flalign*}
                              \end{flalign*}


                            \end{proof}

                            \begin{problem}{9}
                            $\newline$
                            If an integer greater than 4 is a perfect square, then the immediately preceding integer is not prime
                            \end{problem}

                            \begin{proof}

                            \end{proof}

                            \begin{problem}{23}
                            $\newline$
                            The product of any even integer and any integer is even.
                            \end{problem}

                            \begin{proof}

                            \end{proof}


                            % --------------------------------------------------------------
                            %     You don't have to mess with anything below this line.
                            % --------------------------------------------------------------
                            \end{document}
