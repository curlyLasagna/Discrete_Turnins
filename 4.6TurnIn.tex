% --------------------------------------------------------------
% This is all preamble stuff that you don't have to worry about.
% Head down to where it says "Start here"
% --------------------------------------------------------------

\documentclass[12pt]{article}

\usepackage[margin=1in]{geometry}
\usepackage{amsmath,amsthm,amssymb,scrextend}
\usepackage{fancyhdr}
\usepackage{mathtools}
\usepackage{color, soul}
\pagestyle{fancy}

\DeclarePairedDelimiter\ceil{\lceil}{\rceil}
\DeclarePairedDelimiter\floor{\lfloor}{\rfloor}
\newcommand{\N}{\mathbb{N}}
\newcommand{\Z}{\mathbb{Z}}
\newcommand{\I}{\mathbb{I}}
\newcommand{\R}{\mathbb{R}}
\newcommand{\Q}{\mathbb{Q}}
\renewcommand{\qed}{\hfill$\blacksquare$}
\let\newproof\proof
\renewenvironment{proof}{\begin{addmargin}[1em]{0em}\begin{newproof}}{\end{newproof}\end{addmargin}\qed}
% \newcommand{\expl}[1]{\text{\hfill[#1]}$}

\newenvironment{theorem}[2][Theorem]{\begin{trivlist}
    \item[\hskip \labelsep {\bfseries #1}\hskip \labelsep {\bfseries #2.}]}{\end{trivlist}}
\newenvironment{lemma}[2][Lemma]{\begin{trivlist}
    \item[\hskip \labelsep {\bfseries #1}\hskip \labelsep {\bfseries #2.}]}{\end{trivlist}}
\newenvironment{problem}[2][Problem]{\begin{trivlist}
    \item[\hskip \labelsep {\bfseries #1}\hskip \labelsep {\bfseries #2.}]}{\end{trivlist}}
\newenvironment{exercise}[2][Exercise]{\begin{trivlist}
    \item[\hskip \labelsep {\bfseries #1}\hskip \labelsep {\bfseries #2.}]}{\end{trivlist}}
\newenvironment{reflection}[2][Reflection]{\begin{trivlist}
    \item[\hskip \labelsep {\bfseries #1}\hskip \labelsep {\bfseries #2.}]}{\end{trivlist}}
\newenvironment{proposition}[2][Proposition]{\begin{trivlist}
    \item[\hskip \labelsep {\bfseries #1}\hskip \labelsep {\bfseries #2.}]}{\end{trivlist}}
\newenvironment{corollary}[2][Corollary]{\begin{trivlist}
    \item[\hskip \labelsep {\bfseries #1}\hskip \labelsep {\bfseries #2.}]}{\end{trivlist}}

\begin{document}

% --------------------------------------------------------------
%                         Start here
% --------------------------------------------------------------

\lhead{Math 250}
\chead{Luis Gascon}
\rhead{Section 4.6}

\begin{problem}{20}
$\newline$
For all real numbers $x$ and $y$, $\lceil xy \rceil = \lceil x \rceil \cdot \lceil y \rceil$ \\
\textit{False}
\end{problem}

\begin{proof}
	Let x = 1.2 and y = 1.3
	\begin{align*}
		\ceil*{x \cdot y}     & = \ceil*{1.2 \cdot 1.3}     \\
		\ceil*{1.56}          & = 2                         \\
		\ceil*{x} + \ceil*{y} & = \ceil*{1.2} + \ceil*{1.3} \\
		2 + 2                 & = 4
	\end{align*}
	$2 \ne 4$, therefore the statement is false
\end{proof}

\begin{problem}{24}
$\newline$
For any integer $m$ and any real number $x$, if $x$ is not an integer, then $\lfloor x \rfloor + \lfloor m -x \rfloor = m - 1$
\textit{True}
\end{problem}
\begin{proof}
	Let $m \in \Z$ and $x \in \R \ni x \notin \Z$ \\
	Let n = $\floor*{x}$. Then, by definition of floor, $n \in \Z \ni n \le x < n +1$ \\
	Since $x \notin \Z, x \ne n$, so $n < x < n + 1$\\
	\begin{align*}
		 & m - n > m -x > m - n - 1 &  & \text{by algebra} \\
		% & \floor*{m - x} = m - n - 1                        \\
		% & n + m - n - 1 = m  - 1
	\end{align*}
	$-1,n,m \in \Z$ by closure of integers, so $m-n, m - n - 1 \in \Z$. \\
	Since $m-n$, $m-n-1$ are consecutive integers,
	\begin{align*}
		\floor*{m-x}              & = m - n - 1      &  & \text{by definition of floor} \\
		\floor*{x} + \floor*{m-x} & =  n + m - n - 1 &  & \text{by substitution}        \\
		\floor*{x} + \floor*{m-x} & =m - 1           &  & \text{by algebra}
	\end{align*}
	Therefore, $\lfloor x \rfloor + \lfloor m -x \rfloor = m - 1$
\end{proof}
\pagebreak

\begin{problem}{29}
$\newline$
For any odd integer $n$, \\ \\
\centerline{$\ceil[\Bigg]{\cfrac{n^{2}}{4}} = \cfrac{n^{2}+3}{4}$}
\end{problem}
\textit{True} \\
\addtolength{\jot}{8pt}
\begin{proof}
	Let $n$ be odd \\
	$\exists \, k \in \Z \ni n = 2k+1$ by the definition of odd
	\begin{align*}
		n^{2}            & = (2k+1)^{2}                          &  & \text{by substitution} \\
		n^{2}            & = 4k^{2}+4k+1                         &  & \text{by algebra}      \\
		\cfrac{n^{2}}{4} & = \cfrac{4k^{2}+4k+1 }{4}                                         \\
		\cfrac{n^{2}}{4} & = k^{2}+k+\cfrac{1}{4}                                            \\
		k^{2}            & +k < k^{2}+k+\cfrac{1}{4} < k^{2}+k+1
	\end{align*}
	By definition of ceiling, since $k^{2}+k+1 \in \Z$ and $k^{2}+k \in \Z,$\\
	$ k^{2}+k+1$ and $k^{2}+k$ are consecutive integers \\ \\
	$\ceil[\Bigg]{\cfrac{n^{2}}{4}}$ = $k^{2}+k+1 \qquad$
	\lbrack$k^{2}+k+1$ is the integer above $\cfrac{n^{2}}{4}$\Bigg\rbrack\\
	\begin{align*}
		\cfrac{n^{2} + 3}{4} & = \cfrac{(4k^{2}+4k+1)+3}{4} &  & \text{by substitution} \\
		                     & =\cfrac{4k^{2}+4k+4}{4}      &  & \text{by algebra}      \\
		                     & = k^{2}+k+1
	\end{align*}
	So $\ceil[\Bigg]{\cfrac{n^{2}}{4}} = \cfrac{n^{2}+3}{4} $ by the definition of ceiling
\end{proof}
% --------------------------------------------------------------
%     You don't have to mess with anything below this line.
% --------------------------------------------------------------
\end{document}
