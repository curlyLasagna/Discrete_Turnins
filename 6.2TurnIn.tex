% --------------------------------------------------------------
% This is all preamble stuff that you don't have to worry about.
% Head down to where it says "Start here"
% --------------------------------------------------------------

\documentclass[12pt]{article}

\usepackage[margin=1in]{geometry}
\usepackage{amsmath,amsthm,amssymb,scrextend}
\usepackage{fancyhdr}
\pagestyle{fancy}


\newcommand{\N}{\mathbb{N}}
\newcommand{\Z}{\mathbb{Z}}
\newcommand{\I}{\mathbb{I}}
\newcommand{\R}{\mathbb{R}}
\newcommand{\Q}{\mathbb{Q}}
\renewcommand{\qed}{\hfill$\blacksquare$}
\let\newproof\proof
\renewenvironment{proof}{\begin{addmargin}[1em]{0em}\begin{newproof}}{\end{newproof}\end{addmargin}\qed}
% \newcommand{\expl}[1]{\text{\hfill[#1]}$}

\newenvironment{theorem}[2][Theorem]{\begin{trivlist}
    \item[\hskip \labelsep {\bfseries #1}\hskip \labelsep {\bfseries #2.}]}{\end{trivlist}}
\newenvironment{lemma}[2][Lemma]{\begin{trivlist}
    \item[\hskip \labelsep {\bfseries #1}\hskip \labelsep {\bfseries #2.}]}{\end{trivlist}}
\newenvironment{problem}[2][Problem]{\begin{trivlist}
    \item[\hskip \labelsep {\bfseries #1}\hskip \labelsep {\bfseries #2.}]}{\end{trivlist}}
\newenvironment{exercise}[2][Exercise]{\begin{trivlist}
    \item[\hskip \labelsep {\bfseries #1}\hskip \labelsep {\bfseries #2.}]}{\end{trivlist}}
\newenvironment{reflection}[2][Reflection]{\begin{trivlist}
    \item[\hskip \labelsep {\bfseries #1}\hskip \labelsep {\bfseries #2.}]}{\end{trivlist}}
\newenvironment{proposition}[2][Proposition]{\begin{trivlist}
    \item[\hskip \labelsep {\bfseries #1}\hskip \labelsep {\bfseries #2.}]}{\end{trivlist}}
\newenvironment{corollary}[2][Corollary]{\begin{trivlist}
    \item[\hskip \labelsep {\bfseries #1}\hskip \labelsep {\bfseries #2.}]}{\end{trivlist}}

\begin{document}

% --------------------------------------------------------------
%                         Start here
% --------------------------------------------------------------

\lhead{Math 250}
\chead{Luis Gascon}
\rhead{Section 6.2}

\begin{problem}{11}
$\newline$
Use an element argument to prove each statement. \\
Assume that all sets are subsets of universal set $U$. \\ \\
For all sets $A, B, \text{ and } C,$ \\
$ A \cap (B \setminus  C) \subseteq (A \cap B) \setminus  (A \cap C).$
\end{problem}
\begin{proof}
	Let $A, B \text{ and } C$ be sets. Let $x \in A \cap (B \setminus C)$. \\
	Then by definition of intersection, $x \in A \text{ and } x \in (B \setminus C) $. \\
	$x \in (B \setminus C) $ so $x \in B \text{ and } x \notin C $ by definition of set difference. \\
	Since $x \in A \text{ and } x \in B$, $x \in (A \cap B)$ by definition of intersection. \\
	$x \notin (A \cap C)$ since $x \in A$ and $x \notin C$. \\
	Since $ x \in (A \cap B) \text{ and } x \notin (A \cap C)$ by definition of set difference. \\
	Hence $x \in (A \cap B) \setminus  (A \cap C)$ \\
	$\therefore$ $ A \cap (B \setminus  C) \subseteq (A \cap B) \setminus  (A \cap C).$
\end{proof}
\begin{problem}{29}
$\newline$
Use the element method for proving a set equals the empty set to prove each statement.\\
Assume that all sets are subsets of a universal set $U$. \\ \\
For all sets $A, B, \text{ and } C,$ \\
$(A\setminus C) \cap (B \setminus C) \cap (A\setminus B) = \varnothing$
\end{problem}
\begin{proof}(by contradiction): Let $A, B \text{ and } C$ be sets. \\
	Suppose not. That is, suppose $\exists x \in (A\setminus C) \cap (B \setminus C) \cap (A\setminus B)$ \\
	\begin{tabular}{@{}ll@{}}
		Then $x \in (A \setminus C) \text{ and } x \in (B \setminus C) \text{ and } x \in (A \setminus B)$ & by definition of intersection.   \\
		So $x \in A$ and $x \in B$ and $x \notin C$ and $x \notin B$                                       & by definition of set difference.
	\end{tabular} \\
	$\therefore x \in B \text{ and } x \notin B$, which is a contradiction. \\
	So $(A\setminus C) \cap (B \setminus C) \cap (A\setminus B) = \varnothing$ \\
	$\Rightarrow\!\Leftarrow$
\end{proof}

% --------------------------------------------------------------
%     You don't have to mess with anything below this line.
% --------------------------------------------------------------
\end{document}
