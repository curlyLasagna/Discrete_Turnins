% --------------------------------------------------------------
% This is all preamble stuff that you don't have to worry about.
% Head down to where it says "Start here"
% --------------------------------------------------------------

\documentclass[12pt]{article}

\usepackage[margin=1in]{geometry}
\usepackage{amsmath,amsthm,amssymb,scrextend}
\usepackage{fancyhdr}
\pagestyle{fancy}


\newcommand{\N}{\mathbb{N}}
\newcommand{\Z}{\mathbb{Z}}
\newcommand{\I}{\mathbb{I}}
\newcommand{\R}{\mathbb{R}}
\newcommand{\Q}{\mathbb{Q}}
\renewcommand{\qed}{\hfill$\blacksquare$}
\let\newproof\proof
\renewenvironment{proof}{\begin{addmargin}[1em]{0em}\begin{newproof}}{\end{newproof}\end{addmargin}\qed}
% \newcommand{\expl}[1]{\text{\hfill[#1]}$}

\newenvironment{theorem}[2][Theorem]{\begin{trivlist}
		\item[\hskip \labelsep {\bfseries #1}\hskip \labelsep {\bfseries #2.}]}{\end{trivlist}}
\newenvironment{lemma}[2][Lemma]{\begin{trivlist}
		\item[\hskip \labelsep {\bfseries #1}\hskip \labelsep {\bfseries #2.}]}{\end{trivlist}}
\newenvironment{problem}[2][Problem]{\begin{trivlist}
		\item[\hskip \labelsep {\bfseries #1}\hskip \labelsep {\bfseries #2.}]}{\end{trivlist}}
\newenvironment{exercise}[2][Exercise]{\begin{trivlist}
		\item[\hskip \labelsep {\bfseries #1}\hskip \labelsep {\bfseries #2.}]}{\end{trivlist}}
\newenvironment{reflection}[2][Reflection]{\begin{trivlist}
		\item[\hskip \labelsep {\bfseries #1}\hskip \labelsep {\bfseries #2.}]}{\end{trivlist}}
\newenvironment{proposition}[2][Proposition]{\begin{trivlist}
		\item[\hskip \labelsep {\bfseries #1}\hskip \labelsep {\bfseries #2.}]}{\end{trivlist}}
\newenvironment{corollary}[2][Corollary]{\begin{trivlist}
		\item[\hskip \labelsep {\bfseries #1}\hskip \labelsep {\bfseries #2.}]}{\end{trivlist}}

\begin{document}

% --------------------------------------------------------------
%                         Start here
% --------------------------------------------------------------

\lhead{Math 250}
\chead{Luis Gascon}
\rhead{Section 4.4}

% \maketitle

\begin{problem}{28}
$\newline$
\textbf{Theorem:} For all integers $a,b,$ and $c$, if $a \mid bc$ then $a \mid b$ or $a \mid c$.
\end{problem}

\begin{proof}
	Let $a, b, c \in \Z \ni a \mid bc$ \\
	Since $a \mid bc$, $\exists \; d \in \Z \ni ad = bc$ \\
	\begin{align*}
	\end{align*}
\end{proof}

\begin{problem}{29}
$\newline$
\textbf{Theorem:} For all integers $a$ and $b$, if $a \mid b$ then $a^{2} \mid b^{2}$.
\end{problem}
\begin{proof} Let $a,b \in \Z \ni a$ divides $b$ \\
	By definition of divisibility,
	\begin{center}
		$\exists \; k \in \Z \ni b = ak $
	\end{center}
	\begin{align*}
		b^{2} & = (ak)^{2} &  & \text{by substitution}          \\
		      & = (ak)(ak) &  & \text{by algebra}               \\
		      & = a(kak)   &  & \text{by associative property } \\
	\end{align*}
	Let $t = (kak)$, $t \in \Z$ since $a,k \in \Z$ \\
	The sum and products of integers is an integer. \\
	Therefore, $a^{2} \mid b^{2}$ by definition of divisibility since $b^{2} = at$ where $t \in \Z$ by substitution
\end{proof}

% --------------------------------------------------------------
%     You don't have to mess with anything below this line.
% --------------------------------------------------------------
\end{document}
% --------------------------------------------------------------
% This is all preamble stuff that you don't have to worry about.
% Head down to where it says "Start here"
% --------------------------------------------------------------

\documentclass[12pt]{article}

\usepackage[margin=1in]{geometry}
\usepackage{amsmath,amsthm,amssymb,scrextend}
\usepackage{fancyhdr}
\pagestyle{fancy}


\newcommand{\N}{\mathbb{N}}
\newcommand{\Z}{\mathbb{Z}}
\newcommand{\I}{\mathbb{I}}
\newcommand{\R}{\mathbb{R}}
\newcommand{\Q}{\mathbb{Q}}
\renewcommand{\qed}{\hfill$\blacksquare$}
\let\newproof\proof
\renewenvironment{proof}{\begin{addmargin}[1em]{0em}\begin{newproof}}{\end{newproof}\end{addmargin}\qed}
% \newcommand{\expl}[1]{\text{\hfill[#1]}$}

\newenvironment{theorem}[2][Theorem]{\begin{trivlist}
		\item[\hskip \labelsep {\bfseries #1}\hskip \labelsep {\bfseries #2.}]}{\end{trivlist}}
\newenvironment{lemma}[2][Lemma]{\begin{trivlist}
		\item[\hskip \labelsep {\bfseries #1}\hskip \labelsep {\bfseries #2.}]}{\end{trivlist}}
\newenvironment{problem}[2][Problem]{\begin{trivlist}
		\item[\hskip \labelsep {\bfseries #1}\hskip \labelsep {\bfseries #2.}]}{\end{trivlist}}
\newenvironment{exercise}[2][Exercise]{\begin{trivlist}
		\item[\hskip \labelsep {\bfseries #1}\hskip \labelsep {\bfseries #2.}]}{\end{trivlist}}
\newenvironment{reflection}[2][Reflection]{\begin{trivlist}
		\item[\hskip \labelsep {\bfseries #1}\hskip \labelsep {\bfseries #2.}]}{\end{trivlist}}
\newenvironment{proposition}[2][Proposition]{\begin{trivlist}
		\item[\hskip \labelsep {\bfseries #1}\hskip \labelsep {\bfseries #2.}]}{\end{trivlist}}
\newenvironment{corollary}[2][Corollary]{\begin{trivlist}
		\item[\hskip \labelsep {\bfseries #1}\hskip \labelsep {\bfseries #2.}]}{\end{trivlist}}

\begin{document}

% --------------------------------------------------------------
%                         Start here
% --------------------------------------------------------------

\lhead{Math 250}
\chead{Luis Gascon}
\rhead{Section 4.4}

% \maketitle

\begin{problem}{28}
$\newline$
<<<<<<< HEAD
For all integers $a,b,$ and $c$, if $a \mid bc$ then $a \mid b$ or $a \mid c$. \\
False
\end{problem}
\begin{proof}[Counter-example]
	Let $a=4, \, b=6, \, c=10$
	\begin{flalign*}
		 & \quad a \mid bc = 4 | 60                      \\
		 & \quad 4 \nshortmid 6 \: \& \: 4 \nshortmid 10
	\end{flalign*}
	$\therefore$ by counter-example, the statement is false.
\end{proof}
=======
\textbf{Theorem:} For all integers $a,b,$ and $c$, if $a \mid bc$ then $a \mid b$ or $a \mid c$.
\end{problem}

\begin{proof}
	Let $a, b, c \in \Z \ni a \mid bc$ \\
	Since $a \mid bc$, $\exists \; d \in \Z \ni ad = bc$ \\
	\begin{align*}
	\end{align*}
\end{proof}

>>>>>>> ace8925 (Added .DS store to .gitignore)
\begin{problem}{29}
$\newline$
\textbf{Theorem:} For all integers $a$ and $b$, if $a \mid b$ then $a^{2} \mid b^{2}$.
\end{problem}
\begin{proof} Let $a,b \in \Z \ni a$ divides $b$ \\
<<<<<<< HEAD
	By definition of divisibility, $\exists \; k \in \Z \ni b = ak $
	\begin{align*}
		b^{2} & = a^{2}k^{2} &  & \text{by substitution}
	\end{align*}
	Let $t = (k^{2})$, $t \in \Z$ by the closure of integers by multiplication \\
	Therefore, $a^{2} \mid b^{2}$ by definition of divisibility since $b^{2} = a^{2}t$ where $t \in \Z$ by substitution
=======
	By definition of divisibility,
	\begin{center}
		$\exists \; k \in \Z \ni b = ak $
	\end{center}
	\begin{align*}
		b^{2} & = (ak)^{2} &  & \text{by substitution}          \\
		      & = (ak)(ak) &  & \text{by algebra}               \\
		      & = a(kak)   &  & \text{by associative property } \\
	\end{align*}
	Let $t = (kak)$, $t \in \Z$ since $a,k \in \Z$ \\
	The sum and products of integers is an integer. \\
	Therefore, $a^{2} \mid b^{2}$ by definition of divisibility since $b^{2} = at$ where $t \in \Z$ by substitution
>>>>>>> ace8925 (Added .DS store to .gitignore)
\end{proof}

% --------------------------------------------------------------
%     You don't have to mess with anything below this line.
% --------------------------------------------------------------
\end{document}
