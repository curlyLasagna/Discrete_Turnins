% --------------------------------------------------------------
% This is all preamble stuff that you don't have to worry about.
% Head down to where it says "Start here"
% --------------------------------------------------------------
 
\documentclass[12pt]{article}
 
\usepackage[margin=1in]{geometry} 
\usepackage{amsmath,amsthm,amssymb,scrextend}
\usepackage{fancyhdr}
\pagestyle{fancy}

 
\newcommand{\N}{\mathbb{N}}
\newcommand{\Z}{\mathbb{Z}}
\newcommand{\I}{\mathbb{I}}
\newcommand{\R}{\mathbb{R}}
\newcommand{\Q}{\mathbb{Q}}
\renewcommand{\qed}{\hfill$\blacksquare$}
\let\newproof\proof
\renewenvironment{proof}{\begin{addmargin}[1em]{0em}\begin{newproof}}{\end{newproof}\end{addmargin}\qed}
% \newcommand{\expl}[1]{\text{\hfill[#1]}$}

\newenvironment{theorem}[2][Theorem]{\begin{trivlist}
		\item[\hskip \labelsep {\bfseries #1}\hskip \labelsep {\bfseries #2.}]}{\end{trivlist}}
\newenvironment{lemma}[2][Lemma]{\begin{trivlist}
		\item[\hskip \labelsep {\bfseries #1}\hskip \labelsep {\bfseries #2.}]}{\end{trivlist}}
\newenvironment{problem}[2][Problem]{\begin{trivlist}
		\item[\hskip \labelsep {\bfseries #1}\hskip \labelsep {\bfseries #2.}]}{\end{trivlist}}
\newenvironment{exercise}[2][Exercise]{\begin{trivlist}
		\item[\hskip \labelsep {\bfseries #1}\hskip \labelsep {\bfseries #2.}]}{\end{trivlist}}
\newenvironment{reflection}[2][Reflection]{\begin{trivlist}
		\item[\hskip \labelsep {\bfseries #1}\hskip \labelsep {\bfseries #2.}]}{\end{trivlist}}
\newenvironment{proposition}[2][Proposition]{\begin{trivlist}
		\item[\hskip \labelsep {\bfseries #1}\hskip \labelsep {\bfseries #2.}]}{\end{trivlist}}
\newenvironment{corollary}[2][Corollary]{\begin{trivlist}
		\item[\hskip \labelsep {\bfseries #1}\hskip \labelsep {\bfseries #2.}]}{\end{trivlist}}

\begin{document}

% --------------------------------------------------------------
%                         Start here
% --------------------------------------------------------------

\lhead{Math 250}
\chead{Luis Gascon}
\rhead{Chapter 4.2}

\begin{problem}{2}
$\newline$

\item \textbf{Theorem:} For every integer $m$, if $m$ is even then $3m+5$ is odd
\end{problem}
\begin{proof}
	~\\
	Let $m \in \Z \ni m$ is even \\
	By definition of even, $m=2k$ for some $k \in \Z$
	\begin{align*}
		3m + 5 & =3(2k) + 5   &  & \text{by substitution}    \\
		       & =6k + 5      &  & \text{by algebra}         \\
		       & =2(3k+2) + 1 &  & \text{by factoring out 2}
	\end{align*}
	Let $t=3k+2$ \\
	$t \in \Z$ because $3,2,k \in \Z$ and $\Z$ are closed under sums and products. \\
	Therefore, $3m+5 = 2t$ where $t \in \Z$, so $3m+5$ is odd.

\end{proof}

\begin{problem}{9}
$\newline$

\item \textbf{Theorem:} If an integer greater than 4 is a perfect square, then the immediately preceding integer is not prime
\end{problem}

\begin{proof}
	~\\
	Let $n \in \Z$ with $n > 4 \ni n$ is a perfect square \\
	By definition of a perfect square, $\exists \, n = k^{2} \ni k \in \Z$
	\begin{align*}
		n - 1 & = k^{2} - 1  &  & \text{by substitution} \\
		      & = (k-1)(k+1) &  & \text{by algebra}
	\end{align*}
	\begin{align*}
		 & k^2>4                      &  & \text{by substitution} \\
		 & k>2                        &  & \text{by algebra}      \\
		 & k - 1 > 1 \; \& \; k+1 > 3                             \\
	\end{align*}
	$k^{2}-1$ is the product of two integers greater than 1. \\
	Therefore, the immediately preceding integer is composite by definition of composite.
\end{proof}

\pagebreak

\begin{problem}{23}
$\newline$

\item \textbf{Theorem:} The product of any even integer and any integer is even.\\
True
\end{problem}
\begin{proof}
	~\\
	Let $a \in \Z \ni a$ is even and $b \in \Z$. \\
	By definition of even, $\exists \, n \in \Z \ni a = 2n$ \\ \\
	Then
	\begin{align*}
		ab & = 2n\times b    &  & \text{by substitution}    \\
		   & = 2(n \times b) &  & \text{by factoring out 2} \\
	\end{align*}
	\begin{minipage}{30em}
		Let $t=n\times b$. \\
		Then $t \in \Z$ because the products of integers are integers. \\
		Hence, $ab = 2t$, thus $ab$ is even by definition of even.
	\end{minipage}
\end{proof}


                            % --------------------------------------------------------------
                            %     You don't have to mess with anything below this line.
                            % --------------------------------------------------------------
                            \end{document}
