% --------------------------------------------------------------
% This is all preamble stuff that you don't have to worry about.
% Head down to where it says "Start here"
% --------------------------------------------------------------

\documentclass[12pt]{article}

\usepackage[margin=1in]{geometry}
\usepackage{amsmath,amsthm,amssymb,scrextend}
\usepackage{fancyhdr}
\usepackage[makeroom]{cancel}
\pagestyle{fancy}


\newcommand{\N}{\mathbb{N}}
\newcommand{\Z}{\mathbb{Z}}
\newcommand{\I}{\mathbb{I}}
\newcommand{\R}{\mathbb{R}}
\newcommand{\Q}{\mathbb{Q}}
\renewcommand{\qed}{\hfill$\blacksquare$}
\let\newproof\proof
\renewenvironment{proof}{\begin{addmargin}[1em]{0em}\begin{newproof}}{\end{newproof}\end{addmargin}\qed}
% \newcommand{\expl}[1]{\text{\hfill[#1]}$}

\newenvironment{theorem}[2][Theorem]{\begin{trivlist}
    \item[\hskip \labelsep {\bfseries #1}\hskip \labelsep {\bfseries #2.}]}{\end{trivlist}}
\newenvironment{lemma}[2][Lemma]{\begin{trivlist}
    \item[\hskip \labelsep {\bfseries #1}\hskip \labelsep {\bfseries #2.}]}{\end{trivlist}}
\newenvironment{problem}[2][Problem]{\begin{trivlist}
    \item[\hskip \labelsep {\bfseries #1}\hskip \labelsep {\bfseries #2.}]}{\end{trivlist}}
\newenvironment{exercise}[2][Exercise]{\begin{trivlist}
    \item[\hskip \labelsep {\bfseries #1}\hskip \labelsep {\bfseries #2.}]}{\end{trivlist}}
\newenvironment{reflection}[2][Reflection]{\begin{trivlist}
    \item[\hskip \labelsep {\bfseries #1}\hskip \labelsep {\bfseries #2.}]}{\end{trivlist}}
\newenvironment{proposition}[2][Proposition]{\begin{trivlist}
    \item[\hskip \labelsep {\bfseries #1}\hskip \labelsep {\bfseries #2.}]}{\end{trivlist}}
\newenvironment{corollary}[2][Corollary]{\begin{trivlist}
    \item[\hskip \labelsep {\bfseries #1}\hskip \labelsep {\bfseries #2.}]}{\end{trivlist}}

\begin{document}

% --------------------------------------------------------------
%                         Start here
% --------------------------------------------------------------

\lhead{Math 250}
\chead{Luis Gascon}
\rhead{Section 9.2}

\begin{problem}{40}
$\newline$
Prove that for every integer $n \ge 2, \: P(n+1, \: 3) = n^{3}-n$
\end{problem}

\begin{proof}
    Let \(n \in \Z \ni n \ge 2\)
    \begin{align*}
        P(n+1, \: 3) & = \cfrac{(n+1)!}{((n+1)-3)!}                                                 \\
                     & = \cfrac{(n+1)!}{(n-2)!}                                                     \\
                     & = \cfrac{(n+1)n(n-1)\cancel{(n-2)!}}{\cancel{(n-2)!}} &  & \text{by algebra} \\
                     & = n(n+1)(n-1)                                                                \\
                     & = n^3-n
    \end{align*}
\end{proof}

% --------------------------------------------------------------
%     You don't have to mess with anything below this line.
% --------------------------------------------------------------
\end{document}
