% --------------------------------------------------------------
% This is all preamble stuff that you don't have to worry about.
% Head down to where it says "Start here"
% --------------------------------------------------------------
 
\documentclass[12pt]{article}
 
\usepackage[margin=1in]{geometry} 
\usepackage{amsmath,amsthm,amssymb,scrextend}
\usepackage{fancyhdr}
\pagestyle{fancy}

 
\newcommand{\N}{\mathbb{N}}
\newcommand{\Z}{\mathbb{Z}}
\newcommand{\I}{\mathbb{I}}
\newcommand{\R}{\mathbb{R}}
\newcommand{\Q}{\mathbb{Q}}
\renewcommand{\qed}{\hfill$\blacksquare$}
\let\newproof\proof
\renewenvironment{proof}{\begin{addmargin}[1em]{0em}\begin{newproof}}{\end{newproof}\end{addmargin}\qed}
% \newcommand{\expl}[1]{\text{\hfill{}[#1]}$}
 
\newenvironment{theorem}[2][Theorem]{\begin{trivlist}
    \item[\hskip \labelsep {\bfseries #1}\hskip \labelsep {\bfseries #2.}]}{\end{trivlist}}
    \newenvironment{lemma}[2][Lemma]{\begin{trivlist}
        \item[\hskip \labelsep {\bfseries #1}\hskip \labelsep {\bfseries #2.}]}{\end{trivlist}}
        \newenvironment{problem}[2][Problem]{\begin{trivlist}
            \item[\hskip \labelsep {\bfseries #1}\hskip \labelsep {\bfseries #2.}]}{\end{trivlist}}
            \newenvironment{exercise}[2][Exercise]{\begin{trivlist}
                \item[\hskip \labelsep {\bfseries #1}\hskip \labelsep {\bfseries #2.}]}{\end{trivlist}}
                \newenvironment{reflection}[2][Reflection]{\begin{trivlist}
                    \item[\hskip \labelsep {\bfseries #1}\hskip \labelsep {\bfseries #2.}]}{\end{trivlist}}
                    \newenvironment{proposition}[2][Proposition]{\begin{trivlist}
                        \item[\hskip \labelsep {\bfseries #1}\hskip \labelsep {\bfseries #2.}]}{\end{trivlist}}
                        \newenvironment{corollary}[2][Corollary]{\begin{trivlist}
                            \item[\hskip \labelsep {\bfseries #1}\hskip \labelsep {\bfseries #2.}]}{\end{trivlist}}
                             
                            \begin{document}
                             
                            % --------------------------------------------------------------
                            %                         Start here
                            % --------------------------------------------------------------
                            
                            \lhead{Math 250}
                            \chead{Luis Gascon}
                            \rhead{\today}
                             
                            % \maketitle
                             
                            \begin{problem}{22}
                              $\newline$
                              Prove the statement by the methods of exhaustion \\
                              For each integer \emph{n} with $1 \le n \le 10$, $n^{2}-n+11$ is a prime number
                            \end{problem}
                            \begin{proof}
                              Let $n \in \Z$ with $1 \le n \le 10$
                                \begin{align*}
                                  &1^{2}-1+11=11 &&6^{2}-6+11=41 \\ &2^{2}-2+11=13 &&7^{2}-7+11=53 \\ &3^{2}-3+11=17 &&8^{2}-8+11=67 \\ &4^{2}-4+11=23 &&9^{2}-9+11=83 \\ &5^{2}-5+11=31
                                  &&10^{2}-10+11=101
                                \end{align*}
                                Therefore, every integer between 1 and 10, inclusive, returns a prime number when used as an input in the equation $n^{2}-n+11$
                            \end{proof}

                            \begin{problem}{31b}
                            $\newline$
                            Fill in the blanks in the proof of the theorem \\ \\
                            \textbf{Theorem:} Whenever $n$ is an odd integer, $5n^{2}+7$ is even. \\
                            \textbf{Proof:} Suppose $n$ is any \textit{[particular but arbitrarily chosen]} odd integer. \\
                            \textit{[We must show that $5n^{2}+7$ is even]} \\
                            By definition of odd, $n=\underline{2k+1}$ for some integer $k$. \\ \\
                            Then
                            \begin{align*}
                              5n^{2}+7 &=\underline{5(2k+1)^{2}+7} &&\text{by substitution} \\
                                       &= 5(4k^{2}+4k+1) + 7 \\
                                       &= 20k^{2}+20k+12 \\
                                       &=2(10k^{2}+10k+6) &&\text{by algebra} \\
                            \end{align*}
                            Let $t = \underline{10k^{2}+10k+6}$. Then $t$ is an integer because products and sums of integers are integers. \\
                            Hence $5n^{2}+7=2t$, where $t$ is an integer, and thus \underline{even} by definition of even \textit{[as was to be shown]}.
                            \end{problem}
                            \qed


                            % --------------------------------------------------------------
                            %     You don't have to mess with anything below this line.
                            % --------------------------------------------------------------
                            \end{document}
