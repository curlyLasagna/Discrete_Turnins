\message{ !name(5.3TurnIn.tex)}% --------------------------------------------------------------
% This is all preamble stuff that you don't have to worry about.
% Head down to where it says "Start here"
% --------------------------------------------------------------

\documentclass[12pt]{article}

\usepackage[margin=1in]{geometry}
\usepackage{amsmath,amsthm,amssymb,scrextend}
\usepackage{fancyhdr}
\usepackage{tcolorbox}
\usepackage{graphicx}
\pagestyle{fancy}


\newcommand{\N}{\mathbb{N}}
\newcommand{\Z}{\mathbb{Z}}
\newcommand{\I}{\mathbb{I}}
\newcommand{\R}{\mathbb{R}}
\newcommand{\Q}{\mathbb{Q}}
\renewcommand{\qed}{\hfill$\blacksquare$}
\let\newproof\proof
\renewenvironment{proof}{\begin{addmargin}[1em]{0em}\begin{newproof}}{\end{newproof}\end{addmargin}\qed}
% \newcommand{\expl}[1]{\text{\hfill[#1]}$}

\newenvironment{theorem}[2][Theorem]{\begin{trivlist}
    \item[\hskip \labelsep {\bfseries #1}\hskip \labelsep {\bfseries #2.}]}{\end{trivlist}}
\newenvironment{lemma}[2][Lemma]{\begin{trivlist}
    \item[\hskip \labelsep {\bfseries #1}\hskip \labelsep {\bfseries #2.}]}{\end{trivlist}}
\newenvironment{problem}[2][Problem]{\begin{trivlist}
    \item[\hskip \labelsep {\bfseries #1}\hskip \labelsep {\bfseries #2.}]}{\end{trivlist}}
\newenvironment{exercise}[2][Exercise]{\begin{trivlist}
    \item[\hskip \labelsep {\bfseries #1}\hskip \labelsep {\bfseries #2.}]}{\end{trivlist}}
\newenvironment{reflection}[2][Reflection]{\begin{trivlist}
    \item[\hskip \labelsep {\bfseries #1}\hskip \labelsep {\bfseries #2.}]}{\end{trivlist}}
\newenvironment{proposition}[2][Proposition]{\begin{trivlist}
    \item[\hskip \labelsep {\bfseries #1}\hskip \labelsep {\bfseries #2.}]}{\end{trivlist}}
\newenvironment{corollary}[2][Corollary]{\begin{trivlist}
    \item[\hskip \labelsep {\bfseries #1}\hskip \labelsep {\bfseries #2.}]}{\end{trivlist}}

% https://tex.stackexchange.com/questions/212085/tag-placing-with-amsmath
% Moves the tag to the left or right side. Call \leqnomode or \reqnomode before the equation
\makeatletter
\newcommand{\leqnomode}{\tagsleft@true\let\veqno\@@leqno}
\newcommand{\reqnomode}{\tagsleft@false\let\veqno\@@eqno}
\makeatother
\begin{document}

\message{ !name(5.3TurnIn.tex) !offset(-3) }


% --------------------------------------------------------------
%                         Start here
% --------------------------------------------------------------

\lhead{Math 250}
\chead{Luis Gascon}
\rhead{Section 5.3}
% \begin{problem}{15}
% $\newline$
% \end{problem}
\begin{tcolorbox}[width=\textwidth,title={Problem 15},outer arc=0mm]
	Prove by mathematical induction. \\
	$n(n^{2}+5)\text{ is divisible by 6, for each integer } n \ge 0.$
	\newline
	\begin{proof}(by induction) \\
		Let the property $P(n)$ be the sentence ``$n(n^{2}+5)$ is divisible by 6''.
		\leqnomode
		\begin{align}
			\tag*{$P(n)\longrightarrow$} n(n^{2}+5)\text{ is divisible by 6}.
		\end{align}
		\textbf{Base case;} $P(0)$: ``$0(0^{0}+5)$ is divisible by 6''. True, $6 \mid 0$ since $6 \cdot 0 = 0$. \\
		\textbf{Inductive hypothesis}: Let $k \in \Z \ni k \ge 0$. Assume $P(k)$, that is $6 \mid k(k^{2}+5)$. \\
		Then $\exists \, r \in \Z \ni 6r = k(k^{2}+5)$ by definition of divisibility. \\
		\lbrack \textbf{NTS:} $(k+1)((k+1)^{2}+5) = 6r$\rbrack \
		\begin{align*}
			(k+1)((k+1)^{2}+5) &= (k+1)(k^{2} + 2k + 1 +5) && \text{by algebra}\\
		\end{align*}
	\end{proof}
\end{tcolorbox}

% --------------------------------------------------------------
%     You don't have to mess with anything below this line.
% --------------------------------------------------------------
\end{document}

\message{ !name(5.3TurnIn.tex) !offset(-87) }
