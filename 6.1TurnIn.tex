% --------------------------------------------------------------
% This is all preamble stuff that you don't have to worry about.
% Head down to where it says "Start here"
% --------------------------------------------------------------

\documentclass[12pt]{article}

\usepackage[margin=1in]{geometry}
\usepackage{amsmath,amsthm,amssymb,scrextend}
\usepackage{tcolorbox}
\usepackage{fancyhdr}
\pagestyle{fancy}
\usepackage[mathscr]{euscript}

\newcommand{\N}{\mathbb{N}}
\newcommand{\Z}{\mathbb{Z}}
\newcommand{\I}{\mathbb{I}}
\newcommand{\R}{\mathbb{R}}
\newcommand{\Q}{\mathbb{Q}}
\renewcommand{\qed}{\hfill$\blacksquare$}
\let\newproof\proof
\renewenvironment{proof}{\begin{addmargin}[1em]{0em}\begin{newproof}}{\end{newproof}\end{addmargin}\qed}
% \newcommand{\expl}[1]{\text{\hfill[#1]}$}

\newenvironment{theorem}[2][Theorem]{\begin{trivlist}
    \item[\hskip \labelsep {\bfseries #1}\hskip \labelsep {\bfseries #2.}]}{\end{trivlist}}
\newenvironment{lemma}[2][Lemma]{\begin{trivlist}
    \item[\hskip \labelsep {\bfseries #1}\hskip \labelsep {\bfseries #2.}]}{\end{trivlist}}
\newenvironment{problem}[2][Problem]{\begin{trivlist}
    \item[\hskip \labelsep {\bfseries #1}\hskip \labelsep {\bfseries #2.}]}{\end{trivlist}}
\newenvironment{exercise}[2][Exercise]{\begin{trivlist}
    \item[\hskip \labelsep {\bfseries #1}\hskip \labelsep {\bfseries #2.}]}{\end{trivlist}}
\newenvironment{reflection}[2][Reflection]{\begin{trivlist}
    \item[\hskip \labelsep {\bfseries #1}\hskip \labelsep {\bfseries #2.}]}{\end{trivlist}}
\newenvironment{proposition}[2][Proposition]{\begin{trivlist}
    \item[\hskip \labelsep {\bfseries #1}\hskip \labelsep {\bfseries #2.}]}{\end{trivlist}}
\newenvironment{corollary}[2][Corollary]{\begin{trivlist}
    \item[\hskip \labelsep {\bfseries #1}\hskip \labelsep {\bfseries #2.}]}{\end{trivlist}}
\definecolor{palecornflowerblue}{rgb}{0.67, 0.8, 0.94}
\definecolor{anti-flashwhite}{rgb}{0.95, 0.95, 0.96}
\definecolor{floralwhite}{rgb}{1.0, 0.98, 0.94}
\definecolor{ghostwhite}{rgb}{0.97, 0.97, 1.0}
\begin{document}

% --------------------------------------------------------------
%                         Start here
% --------------------------------------------------------------

\lhead{Math 250}
\chead{Luis Gascon}
\rhead{Section 6.1}

% \vspace{4cm}
\begin{tcolorbox}[width=\textwidth, title={Problem 6a \& 6b}, colback=white]
	\begin{center}
		% In case I want a box around it again. Don't judge
		\begin{tcolorbox}[width=.675\textwidth, colback=white, frame empty, boxrule=0pt]
			\setlength{\abovedisplayskip}{0pt}
			\setlength{\belowdisplayskip}{0pt}
			\begin{align*}
				\text{Let } & A = \{x \in \Z \mid x = 5a + 2 \text{ for some integer } a\},  \\
				            & B = \{y \in \Z \mid y = 10b - 3 \text{ for some integer } b\}, \\
				            & C = \{z \in \Z \mid z = 10c + 7 \text{ for some integer } c\}.
			\end{align*}
		\end{tcolorbox}
	\end{center}
	Prove or disprove each of the following statements. \\
	\textbf{a.} $A \subseteq B$ \textit{False} \\
	\begin{proof}[Disproof]
		$12 \in A$ since $12 = 5(2) + 2$ but $12\notin B$ \\
		since $12=10b-3 \iff 15 = 10b$ for some integer $b$ but $ 10 \nmid 15$.
	\end{proof} \\
	\textbf{b.} $B \subseteq A$ \textit{True} \\
	\begin{proof}
		Let $x\in B$ then $x\in \Z \ni x = 10b-3$ for some integer $b$.
		\begin{align*}
			x & = 10b -3                            \\
			  & = 5(2b-1) +2 &  & \text{by algebra}
		\end{align*}
		Let $a = 2b - 1$. $a \in \Z$ by the closure of integers by addition and multiplication. \\
		So $x = 5a+2$ where $a \in \Z$. \\
		$\therefore x \in A$.
	\end{proof}
\end{tcolorbox}
\begin{tcolorbox}[width=\textwidth, title={Problem 32b}, colback=white]
	$\newline$
	Suppose $X = \{a,b\}$ and $Y = \{x,y\}$. Find $\mathcal{P}(X \times Y)$ \\ \\
	$X \times Y = \{(a,x), (a,y), (b,x), (b,y)\}$
	\begin{flalign*}
		\mathcal{P}(X \times Y) = & \:\{ \emptyset , \: \left\{\left(a,\:x\right)\right\}, \:\left\{\left(a,\:y\right)\right\}, \:\left\{\left(b,\:x\right)\right\},\:\left\{\left(b,\:y\right)\right\}, & \\
		                          & \:\left\{\left(a,\:y\right),\:\left(a,\:x\right)\right\}, \:\left\{\left(b,\:x\right),\:\left(a,\:x\right)\right\},                                                  & \\
		                          & \:\left\{\left(b,\:y\right)\right\}, \:\left(a,\:x\right)\right\}, \:\left\{\left(b,\:x\right),\:\left(a,\:y\right)\right\},                                         & \\
		                          & \:\left\{\left(b,\:y\right), \:\left(a,\:y\right)\right\}, \:\left\{\left(b,\:y\right),\:\left(b,\:x\right)\right\},                                                 & \\
		                          & \:\left\{\left(b,\:x\right),\:\left(a,\:y\right),\:\left(a,\:x\right)\right\},                                                                                       & \\ &\:\left\{\left(b,\:y\right), \:\left(a,\:y\right),\:\left(a,\:x\right)\right\}, &\\
		                          & \:\left\{\left(b,\:y\right),\:\left(b,\:x\right),\:\left(a,\:x\right)\right\},                                                                                       & \\
		                          & \:\left\{\left(b,\:y\right),\:\left(b,\:x\right),\:\left(a,\:y\right)\right\},                                                                                       & \\
		                          & \:\left\{\left(a,\:x\right),\:\left(a,\:y\right),\:\left(b,\:x\right),\:\left(b,\:y\right)\right\}                                                                   & \\
	\end{flalign*}
\end{tcolorbox}

% --------------------------------------------------------------
%     You don't have to mess with anything below this line.
% --------------------------------------------------------------
\end{document}
