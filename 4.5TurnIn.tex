% --------------------------------------------------------------
% This is all preamble stuff that you don't have to worry about.
% Head down to where it says "Start here"
% --------------------------------------------------------------

\documentclass[12pt]{article}

\usepackage[margin=1in]{geometry}
\usepackage{amsmath,amsthm,amssymb,scrextend}
\usepackage{fancyhdr}
\pagestyle{fancy}


\newcommand{\N}{\mathbb{N}}
\newcommand{\Z}{\mathbb{Z}}
\newcommand{\I}{\mathbb{I}}
\newcommand{\R}{\mathbb{R}}
\newcommand{\Q}{\mathbb{Q}}
\renewcommand{\qed}{\hfill$\blacksquare$}
\let\newproof\proof
\renewenvironment{proof}{\begin{addmargin}[1em]{0em}\begin{newproof}}{\end{newproof}\end{addmargin}\qed}
% \newcommand{\expl}[1]{\text{\hfill[#1]}$}

\newenvironment{theorem}[2][Theorem]{\begin{trivlist}
		\item[\hskip \labelsep {\bfseries #1}\hskip \labelsep {\bfseries #2.}]}{\end{trivlist}}
\newenvironment{lemma}[2][Lemma]{\begin{trivlist}
		\item[\hskip \labelsep {\bfseries #1}\hskip \labelsep {\bfseries #2.}]}{\end{trivlist}}
\newenvironment{problem}[2][Problem]{\begin{trivlist}
		\item[\hskip \labelsep {\bfseries #1}\hskip \labelsep {\bfseries #2.}]}{\end{trivlist}}
\newenvironment{exercise}[2][Exercise]{\begin{trivlist}
		\item[\hskip \labelsep {\bfseries #1}\hskip \labelsep {\bfseries #2.}]}{\end{trivlist}}
\newenvironment{reflection}[2][Reflection]{\begin{trivlist}
		\item[\hskip \labelsep {\bfseries #1}\hskip \labelsep {\bfseries #2.}]}{\end{trivlist}}
\newenvironment{proposition}[2][Proposition]{\begin{trivlist}
		\item[\hskip \labelsep {\bfseries #1}\hskip \labelsep {\bfseries #2.}]}{\end{trivlist}}
\newenvironment{corollary}[2][Corollary]{\begin{trivlist}
		\item[\hskip \labelsep {\bfseries #1}\hskip \labelsep {\bfseries #2.}]}{\end{trivlist}}

\begin{document}

% --------------------------------------------------------------
%                         Start here
% --------------------------------------------------------------

\lhead{Math 250}
\chead{Luis Gascon}
\rhead{Section 4.5}

% \maketitle

\begin{problem}{19}
$\newline$
\textbf{Theorem:} For all integers $m$ and $n$, if $m$ and $n$ have the same parity, then $5m+7n$ is even. \\
Divide into two cases: $m$ and $n$ are both even and $m$ and $n$ are both odd.
\end{problem}

\begin{proof} (by cases)
	Let $m,n \in \Z \ni m,n$ have the same parity \\
	$\exists \; k,l \in \Z \ni m=2k \: \text{and} \: n=2l$ or \\
	$\exists \; k,l \in \Z \ni m=2k+1 \: \text{and} \: n=2l+1$ \\ \\
	% $m$ and $n$ are even or $m$ and $n$ are odd \\ \\
	\textbf{Case I} ($m=2k$ and $n=2l$):
	\begin{align*}
		5m + 7n =\; & 5(2k) + 7(2l) &  & \text{by substitution} \\
		=\;         & 10k + 14l     &  & \text{by algebra}      \\
		=\;         & 2(5k + 7l)
	\end{align*}
	Let $t = 5k + 7l$. \\
	$t \in \Z$ since $k,l,5,7 \in \Z$ and the set of integers are closed under sums of products.\\
	So $5m+7n=2t$, where $t \ni \Z$ is even by definition of even.\\ \\
	\textbf{Case II} ($m=2k+1$ and $n=2l+1$):
	\begin{align*}
		5m + 7n =\; & 5(2k+1) + 7(2l+1) &  & \text{by substitution} \\
		=\;         & 10k + 5 + 14l + 7 &  & \text{by algebra}      \\
		=\;         & 10k +14l + 12                                 \\
		=\;         & 2(5k + 7l + 6)
	\end{align*}
	Let $t = 5k + 7l +6$. \\
	$t \in \Z$ since $k,l,5,6,7 \in \Z$ and the set of integers are closed under sum of products. \\
	So $5m+7n=2t$, where $t \ni \Z$ is even by definition of even.\\
\end{proof}
\pagebreak
\begin{problem}{24}
$\newline$
\textbf{Theorem:} For all integers $m$ and $n$, if $m \; mod \; 5 = 2$ and $n \; mod \; 5 = 1$ then $mn \; mod \; 5 = 2$.
\end{problem}
\begin{proof}
	Let $m, n \in \Z \ni m \; mod \; 5 = 2 \: \text{and} \: n \; mod \; 5 = 1$. \\
	By Q.R. Theorem, $\exists \: d,f \in \Z \ni m = 5d+2$ and $n = 5f+1$
	\begin{align*}
		m \times n & = (5d+2)(5f+1)        &  & \text{by substitution} \\
		           & = 25df + 5d + 10f + 2 &  & \text{by algebra}      \\
		           & = 5(5df+d+2f) + 2
	\end{align*}
	Let $t = 5df+d+2f$. $t \in \Z$ since $d,f,5 \in Z$ \\
	The set of integers are closed under sums \& products \\
	Therefore, $mn = 5t + 2$ where $t \in \Z$ and $0 \le 2 < 5$. \\
	So $mn \; mod \; 5 = 2 $ by Q.R Theorem.
\end{proof}

% --------------------------------------------------------------
%     You don't have to mess with anything below this line.
% --------------------------------------------------------------
\end{document}
