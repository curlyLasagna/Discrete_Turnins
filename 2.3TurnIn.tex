% --------------------------------------------------------------
% This is all preamble stuff that you don't have to worry about.
% Head down to where it says "Start here"
% --------------------------------------------------------------
 
\documentclass[12pt]{article}
 
\usepackage[margin=1in]{geometry} 
\usepackage{amsmath,amsthm,amssymb,scrextend}
\usepackage{fancyhdr}
\pagestyle{fancy}

 
\newcommand{\N}{\mathbb{N}}
\newcommand{\Z}{\mathbb{Z}}
\newcommand{\I}{\mathbb{I}}
\newcommand{\R}{\mathbb{R}}
\newcommand{\Q}{\mathbb{Q}}
\renewcommand{\qed}{\hfill}
\let\newproof\proof
\renewenvironment{proof}{\begin{addmargin}[1em]{0em}\begin{newproof}}{\end{newproof}\end{addmargin}\qed}
% \newcommand{\expl}[1]{\text{\hfill[#1]}$}
 
\newenvironment{theorem}[2][Theorem]{\begin{trivlist}
    \item[\hskip \labelsep {\bfseries #1}\hskip \labelsep {\bfseries #2.}]}{\end{trivlist}}
    \newenvironment{lemma}[2][Lemma]{\begin{trivlist}
        \item[\hskip \labelsep {\bfseries #1}\hskip \labelsep {\bfseries #2.}]}{\end{trivlist}}
        \newenvironment{problem}[2][Problem]{\begin{trivlist}
            \item[\hskip \labelsep {\bfseries #1}\hskip \labelsep {\bfseries #2.}]}{\end{trivlist}}
            \newenvironment{exercise}[2][Exercise]{\begin{trivlist}
                \item[\hskip \labelsep {\bfseries #1}\hskip \labelsep {\bfseries #2.}]}{\end{trivlist}}
                \newenvironment{reflection}[2][Reflection]{\begin{trivlist}
                    \item[\hskip \labelsep {\bfseries #1}\hskip \labelsep {\bfseries #2.}]}{\end{trivlist}}
                    \newenvironment{proposition}[2][Proposition]{\begin{trivlist}
                        \item[\hskip \labelsep {\bfseries #1}\hskip \labelsep {\bfseries #2.}]}{\end{trivlist}}
                        \newenvironment{corollary}[2][Corollary]{\begin{trivlist}
                            \item[\hskip \labelsep {\bfseries #1}\hskip \labelsep {\bfseries #2.}]}{\end{trivlist}}
                             
                            \begin{document}
                             
                            % --------------------------------------------------------------
                            %                         Start here
                            % --------------------------------------------------------------
                            
                            \lhead{MAT 250}
                            \chead{Luis Gascon}
                            \rhead{\today}
                             
                            % \maketitle
                             
                            \begin{problem}{42}
                              % You can use theorem, proposition, exercise, or reflection here.  Modify x.yz to be whatever number you are proving
                              % Delete this text and write theorem statement here.
                              % We can draw the sets $\R$, $\Q$, $\I$, $\Z$, and $\N$. Let's assume our problem was: Prove that: $$(\forall x \in \N) \left [\sum_{i = 0}^{n}i = \frac{n(n+1)}{2}\right ]$$
                              $\newline$
                              If the argument is valid, identify the rule of inference that guarantees its validity. \\
                              Otherwise, state whether the converse or the inverse error is made. \\
                            \begin{flalign*}
                            &a.\quad p \vee q \\
                            &b.\quad q \rightarrow r \\
                            &c.\quad p \wedge s \rightarrow t \\
                            &d.\quad \sim r \\
                            &e.\quad \sim q \rightarrow u \wedge s \\
                            &f.\quad \therefore t \\
                            \end{flalign*}
                            \end{problem}
                             
                            \begin{proof}
                            % Note 1: The * tells LaTeX not to number the lines.  If you remove the *, be sure to remove it below, too.
                            % Note 2: Inside the align environment, you do not want to use $-signs.  The reason for this is that this is already a math environment. This is why we have to include \text{} around any text inside the align environment.
                            % Note 3: If you want to include text on the right hand side: &&\text{<text>}
                            $\newline$
                            \begin{flalign*}
                            \text{(1)}\qquad 
                            &\sim r &&\text{by (d)} \\
                            &q \rightarrow r &&\text{by (b)} \\
                            \therefore &\sim q &&\text{by modus tollens} \\
                            \text{(2)}\qquad
                            &\sim q \rightarrow u \wedge s &&\text{by (e)} \\
                            &\sim q &&\text{by (1)} \\
                            \therefore\, &u \wedge s &&\text{by modus ponens} \\
                            \text{(3)}\qquad
                            & u \wedge s &&\text{by (2)} \\
                            \therefore\, & s &&\text{by specialization} \\
                            \text{(4)}\qquad 
                            &p \vee q &&\text{by (a)} \\
                            &\sim q &&\text{by (1)} \\
                            \therefore\, &p &&\text{by elimination} \\
                            \text{(5)}\qquad 
                            &p &&\text{by (3)} \\
                            &s &&\text{by (4)} \\
                            \therefore\, &p \wedge s &&\text{by conjunction} \\
                            \text{(6)}\qquad 
                            &p \wedge s \rightarrow t &&\text{by (c)} \\
                            &p \wedge s &&\text{by (5)} \\
                            \therefore\, &t &&\text{by modus ponens}
                            \end{flalign*}
                            \end{proof}
                             
                            % --------------------------------------------------------------
                            %     You don't have to mess with anything below this line.
                            % --------------------------------------------------------------
                            \end{document}
