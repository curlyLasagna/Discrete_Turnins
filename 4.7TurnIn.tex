% --------------------------------------------------------------
% This is all preamble stuff that you don't have to worry about.
% Head down to where it says "Start here"
% --------------------------------------------------------------

\documentclass[12pt]{article}

\usepackage[margin=1in]{geometry}
\usepackage{amsmath,amsthm,amssymb,scrextend}
\usepackage{fancyhdr}
\pagestyle{fancy}

\newcommand{\N}{\mathbb{N}}
\newcommand{\Z}{\mathbb{Z}}
\newcommand{\I}{\mathbb{I}}
\newcommand{\R}{\mathbb{R}}
\newcommand{\Q}{\mathbb{Q}}
\newcommand{\contra}{\hfill$\Rightarrow\!\Leftarrow$}
\renewcommand{\qed}{\hfill$\blacksquare$}
\let\newproof\proof
\renewenvironment{proof}{\begin{addmargin}[1em]{0em}\begin{newproof}}{\end{newproof}\end{addmargin}\qed}
% \newcommand{\expl}[1]{\text{\hfill[#1]}$}
\newenvironment{contradiction}{\begin{addmargin}[1em]{0em}\begin{newproof}}{\end{newproof}\end{addmargin}\contra\qed}

\newenvironment{theorem}[2][Theorem]{\begin{trivlist}
    \item[\hskip \labelsep {\bfseries #1}\hskip \labelsep {\bfseries #2.}]}{\end{trivlist}}
\newenvironment{lemma}[2][Lemma]{\begin{trivlist}
    \item[\hskip \labelsep {\bfseries #1}\hskip \labelsep {\bfseries #2.}]}{\end{trivlist}}
\newenvironment{problem}[2][Problem]{\begin{trivlist}
    \item[\hskip \labelsep {\bfseries #1}\hskip \labelsep {\bfseries #2.}]}{\end{trivlist}}
\newenvironment{exercise}[2][Exercise]{\begin{trivlist}
    \item[\hskip \labelsep {\bfseries #1}\hskip \labelsep {\bfseries #2.}]}{\end{trivlist}}
\newenvironment{reflection}[2][Reflection]{\begin{trivlist}
    \item[\hskip \labelsep {\bfseries #1}\hskip \labelsep {\bfseries #2.}]}{\end{trivlist}}
\newenvironment{proposition}[2][Proposition]{\begin{trivlist}
    \item[\hskip \labelsep {\bfseries #1}\hskip \labelsep {\bfseries #2.}]}{\end{trivlist}}
\newenvironment{corollary}[2][Corollary]{\begin{trivlist}
    \item[\hskip \labelsep {\bfseries #1}\hskip \labelsep {\bfseries #2.}]}{\end{trivlist}}

\begin{document}

% --------------------------------------------------------------
%                         Start here
% --------------------------------------------------------------

\lhead{Math 250}
\chead{Luis Gascon}
\rhead{Section 4.7}

\begin{problem}{13}
$\newline$
S = The product of any irrational number and any nonzero rational number is irrational \\
a. Write a negation for S. \\
b. Prove S by contradiction.
\end{problem}
$\exists \; d \notin \Q \text{ and } f \in \Q \ni d \cdot f \in \Q$ \\
\begin{contradiction}{(by contradiction)}
	Suppose not. \\
	Then, there exists an irrational number, call it $d$ and a nonzero rational number, call it $f$ such that the product of $d$ and $f$ is a rational.     \\
	By definition of rational, \\ $f = h / j$ and $d \cdot f = k / l$ for some integers $h, j, k,$ and $l$ with $h=j\ne0$ and $l\ne0$ since $f$ is a nonzero rational number. \\
	\begin{align*}
		d \cdot \cfrac{h}{j} & = \cfrac{k}{l}   &  & \text{by substitution} \\
		d                    & = \cfrac{kj}{lh} &  & \text{by algebra}      \\
	\end{align*}
	$lh\ne0$ by zero-product property so $d$ is a ratio of integers with a nonzero denominator.    \\
	So $d \in \Q$, hence the supposition is false and the given statement is true
\end{contradiction}
\begin{problem}{22}
$\newline$
\textbf{Theoerm:} For every real number $r$, if $r^{2}$ is irrational then $r$ is irrational. \\
Write what you would suppose and what you would need to show to prove this statement by: \\ \\
a. contradiction \\
Suppose not. \\
That is suppose $\exists \, r \in \R \ni r^{2} \notin \Q$ and $r \in \Q$. \\
We need to show that $r \in \Q$. \\ \\
b. contraposition \\
Suppose $r \in \Q$. \\
We need to show that $r^{2} \in \Q$.
\end{problem}
\pagebreak
\begin{problem}{24}
$\newline$
Prove by contraposition and contradiction \\
\textbf{Theoerm:} The reciprocal of any irrational number is irrational.
\end{problem}
\begin{contradiction}{(by contradiction)}
	Suppose not. \\
	Then $\exists \, k \notin \Q \ni \cfrac{1}{k} \in \Q$ \\
	By definition of a rational, $1/k = a/b \ni a,b \in \Z$ with $b\ne0$
	\begin{align*}
		\cfrac{1}{k}(b) & = \cfrac{a}{b}(b) &  & \text{by algebra} \\
		\cfrac{b}{k}    & = a                                      \\
		k               & = \cfrac{b}{a}    &  &
	\end{align*}
	\text{Since \textit{b} and \textit{k} are both non zeroes, this implies that \textit{a} is non zero} \\
	$a$ and $b$ are both integers and $k$ is a ratio of integers with a nonzero denominator. \\
	Hence $k$ is a rational. \\
	$\therefore$ the supposition is false and the given statement is true
\end{contradiction} \\
\begin{proof}{(by contraposition)} \\
	Let $r \in \R \ni \frac{1}{r} \in \Q$ \\
	By definition of rational, $r = \cfrac{a}{b}$ for some integer $a$ and $b$ with $b\ne0$ \\
	\begin{align*}
		\cfrac{1}{r} & = \cfrac{1}{a/b}                        \\
		r            & = \cfrac{b}{a}   &  & \text{by algebra}
	\end{align*}
	$a$ and $b$ are both integers and $r$ is a ratio of integers with a nonzero denominator, \\
	Hence $r$ is a rational. \\
\end{proof}

% --------------------------------------------------------------
%     You don't have to mess with anything below this line.
% --------------------------------------------------------------
\end{document}
