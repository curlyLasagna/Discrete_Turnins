% --------------------------------------------------------------
% This is all preamble stuff that you don't have to worry about.
% Head down to where it says "Start here"
% --------------------------------------------------------------
 
\documentclass[12pt]{article}
 
\usepackage[margin=1in]{geometry} 
\usepackage{amsmath,amsthm,amssymb,scrextend}
\usepackage{fancyhdr}
\pagestyle{fancy}

 
\newcommand{\N}{\mathbb{N}}
\newcommand{\Z}{\mathbb{Z}}
\newcommand{\I}{\mathbb{I}}
\newcommand{\R}{\mathbb{R}}
\newcommand{\Q}{\mathbb{Q}}
\renewcommand{\qed}{\hfill$\blacksquare$}
\let\newproof\proof
\renewenvironment{proof}{\begin{addmargin}[1em]{0em}\begin{newproof}}{\end{newproof}\end{addmargin}\qed}
% \newcommand{\expl}[1]{\text{\hfill[#1]}$}
 
\newenvironment{theorem}[2][Theorem]{\begin{trivlist}
    \item[\hskip \labelsep {\bfseries #1}\hskip \labelsep {\bfseries #2.}]}{\end{trivlist}}
    \newenvironment{lemma}[2][Lemma]{\begin{trivlist}
        \item[\hskip \labelsep {\bfseries #1}\hskip \labelsep {\bfseries #2.}]}{\end{trivlist}}
        \newenvironment{problem}[2][Problem]{\begin{trivlist}
            \item[\hskip \labelsep {\bfseries #1}\hskip \labelsep {\bfseries #2.}]}{\end{trivlist}}
            \newenvironment{exercise}[2][Exercise]{\begin{trivlist}
                \item[\hskip \labelsep {\bfseries #1}\hskip \labelsep {\bfseries #2.}]}{\end{trivlist}}
                \newenvironment{reflection}[2][Reflection]{\begin{trivlist}
                    \item[\hskip \labelsep {\bfseries #1}\hskip \labelsep {\bfseries #2.}]}{\end{trivlist}}
                    \newenvironment{proposition}[2][Proposition]{\begin{trivlist}
                        \item[\hskip \labelsep {\bfseries #1}\hskip \labelsep {\bfseries #2.}]}{\end{trivlist}}
                        \newenvironment{corollary}[2][Corollary]{\begin{trivlist}
                            \item[\hskip \labelsep {\bfseries #1}\hskip \labelsep {\bfseries #2.}]}{\end{trivlist}}
                             
                            \begin{document}
                             
                            % --------------------------------------------------------------
                            %                         Start here
                            % --------------------------------------------------------------
                            
                            \lhead{Math 250}
                            \chead{Luis Gascon}
                            \rhead{\today}
                             
                            % \maketitle
                             
                            \begin{problem}{1.a}
                              $\newline$
                              % You can use theorem, proposition, exercise, or reflection here.  Modify x.yz to be whatever number you are proving
                            % Delete this text and write theorem statement here. We can draw the sets $\R$, $\Q$, $\I$, $\Z$, and $\N$. Let's assume our problem was: Prove that: $$(\forall x \in \N) \left [\sum_{i = 0}^{n}i = \frac{n(n+1)}{2}\right ]$$
                            \end{problem}
                             
                            \begin{proof}
                            I will induct on $n$ \\
                            %Note 1: The * tells LaTeX not to number the lines.  If you remove the *, be sure to remove it below, too.
                            %Note 2: Inside the align environment, you do not want to use $-signs.  The reason for this is that this is already a math environment. This is why we have to include \text{} around any text inside the align environment.
                            \textbf{Base case (n = 1):  } $\sum_{i=0}^{1}{i} = 1 = \frac{1(1+1)}{2} = 1$ \\
                            \textbf{Inductive Hypothesis: } Assume $\sum_{i = 0}^{k}i = \frac{k(k+1)}{2}$ for some $k\in\N$ \\
                            \textbf{Inductive Step: } [I must show: $\sum_{i = 0}^{k+1}i = \frac{(k+1)(k+2)}{2}$]
                            \begin{flalign*}
                            \sum_{i = 0}^{k+1}i &= k+1 + \sum_{i = 0}^{k}i &&\text{[By definition of series]} \\
                            &= (k + 1) + \frac{k(k+1)}{2} &&\text{[By I.H]} \\
                            & = \frac{(2k+2) + (k^2+k)}{2} \\
                            & = \frac{k^2 + 3k + 2}{3} \\ 
                            & = \frac{(k+1)(k+1)}{2}
                            \end{flalign*}
                            $\therefore$ By the principle of induction, the claim holds for all $n\in\N$
                            \end{proof}
                             
                            \begin{proposition}{x.yz}
                            Let $n\in \Z$.  
                            \end{proposition}
                             
                            \begin{proof}[Disproof]
                              % Whatever you put in the square brackets will be the label for the block of text to follow in the proof environment.
                            % Blah, blah, blah.  I'm so smart.
                            \end{proof}
                             
                            % --------------------------------------------------------------
                            %     You don't have to mess with anything below this line.
                            % --------------------------------------------------------------
                            \end{document}
