% --------------------------------------------------------------
% This is all preamble stuff that you don't have to worry about.
% Head down to where it says "Start here"
% --------------------------------------------------------------

\documentclass[12pt]{article}

\usepackage[margin=1in]{geometry}
\usepackage{amsmath,amsthm,amssymb,scrextend}
\usepackage{fancyhdr}
\usepackage{tabulary}
\usepackage{setspace}
\pagestyle{fancy}


\newcommand{\N}{\mathbb{N}}
\newcommand{\Z}{\mathbb{Z}}
\newcommand{\I}{\mathbb{I}}
\newcommand{\R}{\mathbb{R}}
\newcommand{\Q}{\mathbb{Q}}
\renewcommand{\qed}{\hfill$\blacksquare$}
\let\newproof\proof
\renewenvironment{proof}{\begin{addmargin}[1em]{0em}\begin{newproof}}{\end{newproof}\end{addmargin}\qed}
% \newcommand{\expl}[1]{\text{\hfill[#1]}$}

\newenvironment{theorem}[2][Theorem]{\begin{trivlist}
		\item[\hskip \labelsep {\bfseries #1}\hskip \labelsep {\bfseries #2.}]}{\end{trivlist}}
\newenvironment{lemma}[2][Lemma]{\begin{trivlist}
		\item[\hskip \labelsep {\bfseries #1}\hskip \labelsep {\bfseries #2.}]}{\end{trivlist}}
\newenvironment{problem}[2][Problem]{\begin{trivlist}
		\item[\hskip \labelsep {\bfseries #1}\hskip \labelsep {\bfseries #2.}]}{\end{trivlist}}
\newenvironment{exercise}[2][Exercise]{\begin{trivlist}
		\item[\hskip \labelsep {\bfseries #1}\hskip \labelsep {\bfseries #2.}]}{\end{trivlist}}
\newenvironment{reflection}[2][Reflection]{\begin{trivlist}
		\item[\hskip \labelsep {\bfseries #1}\hskip \labelsep {\bfseries #2.}]}{\end{trivlist}}
\newenvironment{proposition}[2][Proposition]{\begin{trivlist}
		\item[\hskip \labelsep {\bfseries #1}\hskip \labelsep {\bfseries #2.}]}{\end{trivlist}}
\newenvironment{corollary}[2][Corollary]{\begin{trivlist}
		\item[\hskip \labelsep {\bfseries #1}\hskip \labelsep {\bfseries #2.}]}{\end{trivlist}}

\begin{document}

% --------------------------------------------------------------
%                         Start here
% --------------------------------------------------------------

\lhead{Math 250}
\chead{Luis Gascon}
\rhead{Section 4.3}
% \maketitle

\begin{problem}{23}
$\newline$
True or false? \\ If $k$ is any even integer and $m$ is any odd integer, then ${(k+2)}^{2} - {(m-1)}^{2}$ is even. Explain. \\ \\ False.
\\ \\
\begin{tabulary}{\textwidth }{@{} J J @{}}
	1. The sum, product, and difference of any two even integers are even.
	&5. The sum of any odd integer and any even integer is odd.          \\ \\
	2. The sum and difference of any two odd integers are even.
	&6. The difference of any odd integer minus any even integer is odd. \\ \\
	3. The product of any two odd integers is odd.
	&7. The difference of any even integer minus any odd integer is odd. \\
	4. The product of any even integer and any odd integer is even.
\end{tabulary}
\end{problem}
\begin{proof}
	Let $k$ be an even integer and $m$ be an odd integer.
	\begin{itemize}
		\item $k+2$ is even since $k$ and 2 are even and the sum of even integers is even by 1.)
		\item $(k+2)^{2}$ is even since $k+2$ is even and the product of even integers is even by 1.)
		\item $m-1$ is even since $m$ and 1 are odd and the sum and difference of any two odd integers are even by 2.)
		\item $(m-1)^{2}$ is odd since $m-1$ is odd and the product of odd integers is odd by 3.)
		\item ${(k+2)}^{2} - {(m-1)}^{2}$ is odd since $(k+2)^{2}$ is even and $(m-1)^{2}$ is odd and the difference of any even integer minus any odd integer is odd by 7.)
		      % The sum and difference of any two odd integers are even by 2.) \\
	\end{itemize}
\end{proof}

% --------------------------------------------------------------
%     You don't have to mess with anything below this line.
% --------------------------------------------------------------
\end{document}
